\documentclass[12pt]{beamer}

\usepackage[utf8]{inputenc}
\usepackage[francais]{babel}
\usepackage{amsmath,amsfonts,amssymb,amsthm}
\usepackage{enumerate}
\usepackage{graphicx}
\usepackage{tikz}

\usetheme{Boadilla}

\begin{document}
	\author{Mich-Mich}
	\title{Un petit exemple de Beamer}
	\institute{Université de Mons}
	\frame{\maketitle}
	\begin{frame}
		\frametitle{Table des matières}
		\begin{enumerate}
			\item Énumérations
			\item Blocs
			\item Sur deux colonnes
		\end{enumerate}
	\end{frame}
	\begin{frame}
		\frametitle{Le master en mathématiques à l'UMONS}
		Les finalités du master en mathématiques
		\begin{enumerate}
			\item finalité approfondie
			\item finalité spécialisée
			\item finalité didactique
		\end{enumerate}
		Les programmes
		\begin{itemize}
			\item Tronc commun
			\item Cours et stages de finalité
			\item Mémoire
			\item Projet intégré
			\item Cours au grand choix
		\end{itemize}
	\end{frame}
	\begin{frame}
		\frametitle{La Chat}
		\begin{alertblock}{Le chat}	
			Le Chat est une série de bande dessinée créée par Philippe Geluck qui met en scène le héros éponyme dans de nombreux gags.
		\end{alertblock}
		\begin{exampleblock}{Exemple de phrases}
			\begin{itemize}
				\item Pour voir qu'il fait noir, on n'a pas besoin de lumière.
				\item Si les lentilles vous font péter, portez des lunettes.
				\item Le coup du lapin ça doit être terrible chez la girafe.
				\item Je préfère le vin d'ici à l'eau de là.
				\item Contrairement à ce que l'on pourrait croire, Germinal n'est pas une ouvre mineure.
				\item Au pays des cyclopes, les borgnes sont aveugles.
			\end{itemize}
		\end{exampleblock}
	\end{frame}
	\begin{frame}
		\frametitle{De quoi passer son temps...}
	\end{frame}
	\begin{frame}
		\frametitle{Avec une image maintenant}
		\begin{figure}[b]
			\includegraphics[scale=0.6,angle=90]{umons.png}
		\end{figure}
	\end{frame}
	\begin{frame}
		\frametitle{Et du code}
	\end{frame}
	\begin{verbatim}
	var i = 0;
	var max = 0;
	for(i in tab) {
	if(i > max) {
	max = i;
	}
	}
	return max;
	\end{verbatim}
\end{document}